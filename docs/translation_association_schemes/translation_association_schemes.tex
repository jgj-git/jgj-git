\documentclass[a4paper,12pt]{article}
\usepackage[utf8]{inputenc}
\usepackage[T1]{fontenc}
\usepackage[style=english]{csquotes}
\usepackage[spanish,catalan,english]{babel}
\usepackage{graphicx}
\usepackage{titlesec}
\usepackage{geometry}
\usepackage{hyperref}
\usepackage{fancyhdr}
\usepackage{amsthm}
\usepackage{amsmath}
\usepackage{amssymb}
\usepackage{ mathrsfs }
\usepackage{verbatim}
\usepackage[toc,page]{appendix}
\usepackage[table,xcdraw]{xcolor}

%% Bibliography
\usepackage[sorting=none]{biblatex}
%\DefineBibliographyStrings{english}{
%  bibliography = {References},
%}
\addbibresource{My_Library.bib}
% If you want to break on URL numbers
\setcounter{biburlnumpenalty}{9000}
% If you want to break on URL lower case letters
\setcounter{biburllcpenalty}{9000}
% If you want to break on URL UPPER CASE letters
\setcounter{biburlucpenalty}{9000}

% Custom enumerate
\usepackage{enumitem}

% gaussian binomial coefficient
\newcommand{\gaussian}{\genfrac{[}{]}{0pt}{}}

% horizontal line for matrix
\newcommand*{\horzbar}{\rule[.5ex]{2.5ex}{0.5pt}}

%amsmath enviroments
\theoremstyle{plain}
\newtheorem{theorem}{Theorem}
\newtheorem{proposition}[theorem]{Proposition}
\newtheorem{lemma}[theorem]{Lemma}
\newtheorem{corollary}[theorem]{Corollary}

\renewcommand{\qedsymbol}{$\blacksquare$}

\theoremstyle{definition}
\newtheorem{definition}[theorem]{Definition}
%\newtheorem{definition}{Definition}[section]
\newtheorem{example}[theorem]{Example}

\theoremstyle{remark}
\newtheorem{remark}[theorem]{Remark}

%paragraph spacing and indent
\setlength{\parskip}{0pt}
\setlength{\parindent}{15pt} %default is 15

%equation numbering
%\renewcommand{\theequation}{\arabic{section}.\arabic{equation}}
\makeatletter
\renewcommand\theequation{\arabic{equation}}
\makeatother

%% Assemble the document
\begin{document}

\begin{titlepage}
    \begin{center}
        \vspace{1cm}
        \Huge
        \textbf{Translation Association Schemes}

        \vspace{0.5cm}
        \LARGE
        A brief introduction

        \vspace{1.5cm}

       \textbf{Joel González}

       \vfill
    \end{center}
\end{titlepage}

\pagenumbering{roman}
\clearpage
\pagenumbering{arabic}

\section{Introduction}

This document attemps to provide an introduction to the topic of translation
association schemes. During my bachelor's thesis, which dealt with rank-metric
codes, I had to learn from scratch all of the main identities and
theorems of translation association schemes, and I found that there was not much
content available on the internet. Furthermore, some of the main sources regarding
them (which are cited throughout this document) were very hard to come by, and
I could only get access to them researchers at my university who happened to have them.
The content presented here is mostly self-contained,
only requiring basic concepts of group theory and linear algebra which anyone
studying a bachelor's in mathematics will be familiar with; hopefully it will
be useful to the reader who is not yet familiar with association schemes and
wants a quick but detailed overview of the properties of translation association
schemes, with all the necessary proofs.

Association schemes are one of the more relevant objects of algebraic combinatorics.
They are no more than special partitions of finite sets, but their particular
structure has been studied thoroughly and a there is a very rich set of results
and identities.

We start by defining association schemes.
\begin{definition}
    \label{def:association_scheme_partition}
    Let $ X $ be a finite set. An \textbf{association scheme} with $ d $ classes is a
    pair $ (X, \mathcal{R}) $ such that
    \begin{enumerate}[label=(\roman*)]
        \item $ \mathcal{R} = \{ R_0, R_1, \dots, R_d \} $ is a partition of
            $ X \times X $;
        \item $ R_0 = \Delta := \{ (x,x) \mid x \in X \} $;
        \item \label{as_sym}
            For any $ i \in \{ 0,1,\dots,d \} $, $ R_i^T := \{ (y,x) \mid
            (x,y) \in R_i \} $ is a class in $ \mathcal{R} $;
        \item For any $ i,j,k \in \{ 0,1,\dots,d \} $ there is a constant,
            $ p_{ij}^k $, such that, for any $ (x,y) \in R_k $, $ p_{ij}^k $ counts
            the number of $ z \in X $ such that $ (x,z) \in R_i $ and
            $ (z,y) \in R_j $;
        \item \label{as_com}
            For all $ i,j,k \in \{ 0,1,\dots, d \} $, $ p_{ij}^k = p_{ji}^k $.
    \end{enumerate}
\end{definition}
Some authors do not assume property \ref{as_com}, refer association schemes with
this property as commutative association schemes. In this section we will assume
all association schemes to be commutative.

In coding theory, we are usually working with association schemes
where the relation is symmetric:
\begin{definition}
    \label{def:sym_association_scheme_partition}
    We say that an association scheme is \textbf{symmetric} if, instead of \ref{as_sym}, it
    holds that
    \[ R_i = R_i^T, \forall i \in \{ 0,1,\dots,d \} \]
\end{definition}
Unless specified otherwise, any association schemes mentioned from now on will
be symmetric\footnote{For a very thorough introduction to non-symmetric association
schemes and related algebraic concepts, see University of Waterloo's notes from
Godsil \cite{godsil_online}}. Note that for symmetric association schemes, \ref{as_com}
is not an axiom but rather a consequence of all the other axioms.

\section{The theory of association schemes}
In this section, we will go over the existence of the
basis of idempotent matrices in the Bose-Mesner algebra. To do this, we will
be mostly following Godsil's \parencite{godsil_erdoskorado_2015} which, instead of
relying on
know results of linear algebra, gives a more combinatorial construction of the
basis of idempotents. Brouwer et al.'s \parencite{brouwer_distance-regular_1989} and
and Van Lint and Wilson's \parencite{van_lint_course_2001} were also consulted to
obtain some intermediate results.

In symmetric association schemes, the number $ n_i := p_{ii}^0 $, called the
\textbf{valency} of $ R_i $, is the number of $ z \in X $ such that
$ (x,z) \in R_i $. Therefore, it holds that
\[ n := \mid X \mid = \sum_{ i=0 }^{ d } n_{ii}^0 \]
Note that $ n_0 = 1 $.

We now state some useful properties of the coefficients of association schemes.
\begin{lemma}[{\parencite[Lemma 2.1.1]{brouwer_distance-regular_1989}}]
    The parameters $ n_i$ and $ p_{ij}^k $ of an association scheme with $ d $
    classes satisfy the following:
    \begin{enumerate}[label=(\roman*)]
        \item $ p_{0j}^k = \delta_{jk} $,
        \item $ p_{ij}^0 = \delta_{ij}n_j $,
        \item $ p_{ij}^k = p_{ji}^k $,
    \end{enumerate}
    \begin{proof}
        \begin{enumerate}[label=(\roman*)]
        \item Note that $ (x,z) \in R_0 \iff x = z $. If $ j = k $, then it is
            clear that $ p_{0j}^k = 0 $. If $ j \neq 0 $, then there is one
            $ (x,y) \in R_k - R_j $, so for that $ (x,y) $ there are no $ z\in X $
            such that $ (x,z) = (x,x) \in R_0 $ and $ (z,y) = (x,y) \in R_j $,
            therefore $ p_{0j}^k $.
        \item If $ i = j $, by definition $ p_{jj}^0 = n_j $. Suppose
            $ i \neq j $ and $ p_{ij}^k >= 1 $. Then, for every $ x \in X $,
            $ (x,x) \in R_0 $ and there is at least one $ z \in X $ such that
            $ (x,z) \in R_i \iff (z, x) \in R_i $ and $ (z,x) \in R_j $, so
            $ (z,x) \in R_i \cap R_j $, which is a contradiction with the fact
            that $ \mathcal{R} $ is a partition of $ X \times X $.
        \item Given $ i,j,k \in \{ 0,1,\dots,d \} $, for any $ (x,y) \in R_k
            \iff (y,x) \in R_k $,
            $ (x,z) \in R_i \iff (z,x) \in R_i $ and $ (z,y) \in R_j \iff
            (y,z) \in R_j$, so it is clear that $ p_{ij}^k = p_{ji}^k $.
             \end{enumerate}
    \end{proof}

\end{lemma}

One can view an association scheme as a complete graph with labeled edge:
the vertices are the elements in $ X $ and there is an edge with label $ i $
between two vertices $ x,y \in X $ if and only if $ (x,y) \in R_i $. The subgraph
obtained by only selecting the edges labeled $ i $ is usually denoted by
$ X_i $.

We define the \textbf{adjancecy matrices} of an association scheme as the association
matrices of each subgraph $ X_i $, that is,
\[
    \begin{cases}
        (A_i)_{xy} = 1 &\text{ if } (x,y) \in R_i \\
        0 &\text{ otherwise}
    \end{cases}
\]
These are clearly $ n \times n $ symmetric matrices, and
\[ A_0 = I \]
\[ A_0 + A_1 + \dots + A_d = J \]
where $ I $ is the $ n \times n $ identity matix and $ J $ is the $ n \times n $
matrix with all coefficients $ 1 $. This induces an equivalent definition of
association schemes in terms of matrices, which is also commonly used in the
literature (\parencite[Chapter 3]{godsil_erdoskorado_2015}).
\begin{definition}
    A (commutative, not necessarily symmetric) association scheme is a set of
    $ n \times n $ $ \{ 0,1 \} $-matrices $ \mathcal{A} = \{ A_0, A_1, \dots, A_d \} $
    with the following properties:
    \begin{enumerate}[label=(\roman*)]
        \item $ \sum_{ i=1 }^{ k } = J $;
        \item $ A_0 = I $;
        \item \label{as_mat_rel} $ A_i^T \in \mathcal{A} $ for all
            $ i \in \{ 0,1, \dots, d \} $;
        \item for all $ i,j \in \{ 0,1, \dots, d \} $, the matrix product
            $ A_i A_j $ is in the span of $ \mathcal{A} $;
        \item $ A_i A_j = A_j A_i $ for all $ i $ and $ j $.
    \end{enumerate}
\end{definition}
It is immediate that the first three axioms of this definition are equivalent
to the first three in the definition based on relations. The last two are also
equivalent:
\begin{proof}
    We define $ p_{ij}^k $ as the coefficient of $ A_k $ when we express $ A_iA_j $
    as an element in the span of $ \mathcal{A} $, that is,
    \[ A_iA_j = \sum_{ k=0 }^{ d } p_{ij}^k A_k \]
    We now prove that this definition is equivalent to the one given previously.
    Note that $ (A_i A_j)_{xy} $ is the number of $ z \in X $ such that
    $ (A_i)_{xz} = 1 $ and $ (A_j)_{zy} = 1 $. From \ref{as_mat_rel}, we deduce
    that, for every $ x,y $ and every $ i,j \in \{ 0,1,\dots,d \} $, either
    $ (A_i)_{xy} = 0 $ or $ (A_j)_{xy} = 0 $. Therefore, fixing $ k,i,j
    \in \{ 0,1,\dots,d \}$, it is clear that for any $ (A_k)_{xy} $ the number
    of $ z \in X $ such that $ (A_i)_{xz} $ and $ (A_j)_{zy} $ is precisely
    the coefficient $ p_{ij}^k $ and that this coefficient is independent of
    the choice of $ x,y $.
\end{proof}
The definition of symmetric association schemes based on matrices is analogous.
\\
The set $ \mathcal{A} $ clearly forms an algebra with respect to
the matrix multiplication. Furthermore, $ \mathcal{A} \cup \{ 0 \} $ forms an
algebra with respect to the \textbf{Hadamard} or \textbf{Schur} product, which
is defined as
\[ (A \circ B)_{xy} = A_{xy}B_{xy} \]
This is immediate from the fact that all matrices in the association scheme are
Schur orthogonal and Schur idempotent, that is,
\[
    A_i \circ A_j = \begin{cases}
        A_i &\text{ if } i = j \\
        0 &\text{ otherwise}
    \end{cases}
\]
It is also immediate that the matrices are linearly independent.

We will denote the linear span of $ \mathcal{A} $ over the complex numbers by
$ \mathbb{C}(\mathcal{A}) $. In the literature, $ \mathbb{C}(\mathcal{A}) $ is
often referred to as the \textbf{Bose-Mesner} algebra of $ \mathcal{A} $.

All of this is summed up in the following lemma:

\begin{lemma}
    The Bose-Mesner algebra of an association scheme $ \mathcal{A} $,
    $ \mathbb{C}(\mathcal{A})  $, is a commutative matrix algebra with identity,
    which is closed under transposition and contains the all-ones matrix.
    \label{lm:ekr_bose_mesner}
\end{lemma}

The Bose-Mesner algebra is interesting because a basis of orthogonal idempotent
matrices exists for it. In the literature, this is commonly deduced using known
results of linear algebra, since the fact that the $ A_i $ commute implies that
they can be simultaneously diagonalized (\cite{brouwer_distance-regular_1989},
\cite{van_lint_course_2001}). However, a combinatorial construction of this
basis is provided in \cite{godsil_erdoskorado_2015}, which we will develop in
the next couple of pages.

\begin{theorem}[{\parencite[Theorem 3.4.1]{godsil_erdoskorado_2015}}]
    Let $ M $ be a commutative matrix algebra with identity over $ \mathbb{C} $.
    Assume that, for all $ N \in M $, it holds that $ N^2 = 0 $ only if $ N = 0 $.
    Then, each matrix in $ M $ can be expressed as a linear combination of
    pairwise othogonal independents.
    \label{th:ekr_commutative}
\end{theorem}
\begin{proof}
    Suppose $ A \in M $. Call $ m_A $ the minimal polynomial of $ A $, with
    \[ m_A(x) = \prod_{ i=1 }^{ k } (x-\alpha_i)^{\lambda_i} \]
    We now define
    \[ m_{A,i}(x) = \frac{m_A(x)}{(x-\alpha_i)^{\lambda_i}} \]
    Note that $ m_{A,1}, \dots, m_{A,k} $ are coprime by definition and that they
    are univariate polynomials over a field. Therefore, Bezout's identity states that
    there are $ f_1(x), \dots, f_k(x) \in \mathbb{C}[X] $ such that
    \[ f_1(x)m_{A,1}(x) + \dots + f_k(x)m_{A,k}(x) = 1 \]
    If we replace by $ A $, we find that
    \begin{equation}
        f_1(A)m_{A,1}(A) + \dots + f_k(A)m_{A,k}(A) = I
        \label{bezout_eq}
    \end{equation}
    We will use this identity to find the idempotent matrices. Define
    $ E_i = f_i(A)m_{A,i}(A) $. First of all, we see that they are pairwise orthogonal.
    If $ i \neq j $, then it is clear that $ m_{A,i}m_{A_j} $ is a
    multiple of $ m_A $, so
    \[ m_{A,i}m_{A,j} = 0 \]
    Therefore, $ E_iE_j = 0 $, i.e., they are pairwise othogonal.

    We can use this to see that the $ E_i $ are idempotent. If we multiply
    \ref{bezout_eq} by $ E_i $ for a given $ i \in \{ 1,\dots,k \} $, we get that:
    \[ E_i(E_1 + \dots E_k) = E_iI \iff E_i^2 = E_i \]
    The only thing left to show is that $ A $ is a linear combination of these $ E_i $.
    It is immediate that $ (x-\alpha_i)^{\lambda_i}f_i(x) $ is a multiple of $ m_A(X) $,
    so it holds that
    \[ (A-\alpha_iI)^{\lambda_i}E_i = 0 \]
    Using the fact that $ E_i $ is idempotent,
    \[ ((A-\alpha_iI)E_i)^{\lambda_i} = 0 \]

    Note that, by hypothesis, $ N^2 = 0 \Rightarrow N = 0 $ for every $ N \in M $.
    This also implies that $ N^n = 0 \Rightarrow N = 0 $, which we can prove by
    induction: if we suppose that $ N^{k} = 0 \iff N = 0 $ for all $ 1 <= k <= n-1 $
    \[ N^n = 0 \Rightarrow N^{n + (n\pmod{2})} = 0 \Rightarrow (N^{\frac{n+(n\pmod{2})}{2}})^2 = 0
    \Rightarrow N^{\frac{n+(n\pmod{2})}{2}} = 0 \Rightarrow N = 0 \]

    Therefore, from the previous identity we deduce that $ (A-\alpha_iI)E_i = 0
    \iff A E_i = \alpha_i E_i$. Multiplying \ref{bezout_eq} by $ A $, we now get
    that
    \[ A = AE_i + \dots + AE_k = \alpha_1 E_1 + \dots + \alpha_k E_k \]
    so $ A $ can be expressed as a linear combination of pairwise othogonal
    idempotents. Note that from $ AE_i = \alpha_i E_i $ we get that the columns of
    $ E_i $ are eigenvectors of A with eigenvalue $ \alpha_i $.

\end{proof}

We will now prove that this theorem's hypothesis holds for the Bose-Mesner algebra
of any association scheme.

\begin{lemma}[{\parencite[Lemma 3.4.2]{godsil_erdoskorado_2015}}]
    If $ N $ and $ N^* $ commute, then $ N^2=0 \Rightarrow N=0 $.
    \label{lm:ekr_N2}
\end{lemma}
\begin{proof}
    If $ N $ and $ N^2 $ commute and $ N^2=0 $, then it holds that
    \[ 0 = (N^*)^2 N^2 = (N^*N)^2 \]
    which implies that
    \[ 0 = tr((N^*N)^2) = tr((N^*N)^*(N^*N)) \]
    It is easy to show that $ tr(H*H) = 0 \Rightarrow H = 0 $ for any matrix $ H $
    with coefficients in $ \mathbb{C} $. Note that
    \[ (H^*H)_{ii} = \sum_{ j=1 }^{ n } \overline{H_{ji}}H_{ji} \]
    which is a positive real number. Therefore,
    \[ tr(H^*H) = 0 \Rightarrow \sum_{ j=1 }^{ n } \overline{H_{ji}}H_{ji} = 0,
        \forall i \Rightarrow \overline{H_{ji}}H_{ji} = 0, \forall i,j
        \Rightarrow H_{ji} = 0, \forall i,j \]
    We now use this to deduce that $ N^*N = 0 $. Applying the same argument, we
    conclude that $ N = 0 $.
\end{proof}

One can define a partial ordering on the idempotent matrices of a commutative
algebra. For any two idempotent $ E $ and $ F $, we write $ E \leq F $ if
$ FE = E $. This relation is clearly a partial order:
\begin{enumerate}
    \item Reflexive: $ E^2 = E $ because $ E $ is idempotent.
    \item Transitive: $ FE = E $ and $ GF = F $ implies $ GE = GFE = FE = E $.
    \item Antisymmetric: if $ FE = E $ and $ EF = F $, since they commute we have
        that $ E = FE = EF = F $.
\end{enumerate}
Note that the fact that $ FE = E $ implies that the column space of $ E $ is
a subspace of the colunm space of $ F $:
\begin{proof}
    If $ c = ( c_1,\dots,c_n) $ is a column in $ E $, and $ f_1, \dots, f_n $
    are the columns of $ F $, then
    \[ Fc = c \iff c_1f_1 + \dots + c_kf_k = c \]
    which implies that $ c \in \langle f_1, \dots, f_k \rangle $.
\end{proof}

We will call a \textbf{minimal idempotent} a minimal element of the set of
nonzero idempotent
matrices with respect to this order, i.e., an idempotent matrix $ A $ is minimal
if there is no other non-zero idempotent matrix $ F $, $ F \neq E $, such that
$ F \leq E $.
This concept will be useful in that there is a basis of idempotent matrices for
the Bose-Mesner algebra of any association scheme.

\begin{lemma}[{\parencite[Lemma 3.4.3]{godsil_erdoskorado_2015}}]
    Any set of idempotent matrices in a commutative algebra contains a subset of
    minimal idempotent matrices. Furthermore, these minimal idempotent matrices
    are pairwise orthogonal.
    \label{lm:ekr_minimal}
\end{lemma}
\begin{proof}
    Suppose that $ E $ and $ F $ are distinct idempotents in a commutative algebra,
    $ E \leq F \iff FE = E $. Then,
    \[ F-E \neq 0 \iff F - FE \neq 0 \iff F(I-E) \neq 0 \]
    Since $ E $ is idempotent, we have that
    \[ E(I-E) = 0 \]
    From this and the previous inequality we can deduce that the span of $ E $'s
    columns cannot be the same as the span of $ F $'s columns, and since $ E \leq F $
    we get that the former is strictly contained in the latter. This implies that
    if $ E \leq F $, then the dimension of the vector subspace generated by the columns of
    $ E $ is strictly smaller that the dimension of the vector subspace generated
    by the columns of $ F $. Now, suppose that $ E_1, \dots, E_m $ are distint
    non zero idempotent matrices of a commutative algebra, such that
    \[ E_1 \leq \dots \leq E_m \]
    hen, $ m $ must be bounded by the number of columns in these matrices, i.e.,
    all chains in the partial order contain a finite number of elements and
    so they must have a minimal element. This proves that there is a set of
    minimal idempotent matrices.

    Suppose now that $ E $ and $ F $ are distinct minimal idempotent matrices.
    It is immediate that $ EF \leq E $ and $ FE \leq F $. Therefore, $ EF = FE
    = 0 $, and they are pairwise orthogonal.
\end{proof}

\begin{theorem}[{\parencite[Theorem 3.4.4]{godsil_erdoskorado_2015}}]
    Let $ \mathcal{A} $ be an association scheme and $ \mathbb{C}(\mathcal{A}) $
    its its Bose-Mesner algebra. Then $ \mathbb{C}(\mathcal{A}) $ has a basis of
    pairwise othogonal idempotet matrices $ \{ E_0, \dots, E_d \} $. Furthermore,
    this basis satisfies the following properties:
    \begin{enumerate}[label=(\roman*)]
        \item $ E_0 = \frac{1}{n}J $;
        \item $ \sum_{ i=0 }^{ d } E_i = I $;
        \item $ E_i^T \in \{ E_0, \dots, E_d \} $ for all $ i \in \{ 0,1,\dots,d \} $;
        \item for any $ i,j \in \{ 0,1,\dots,d \} $, $ E_i \circ E_j $ is in
            $ \mathbb{C}(\mathcal{A}) $;
        \item $ E_iE_j = \delta_{ij} $ for all $ i,j \in \{ 0,1,\dots,d \} $.
    \end{enumerate}
    \label{th:idemp_mat_properties}
\end{theorem}
\begin{proof}
    We start by proving the first statement. If $ A \in \mathbb{C}(\mathcal{A}) $,
    then $ A^* \in \mathbb{C}(\mathcal{A}) $ because the matrices
    $ A_0, A_1, \dots, A_d $ have coefficients in $ \{ 0,1 \} $ and
    $ A_i \in \mathcal{A} $ for all $ i \in \{ 0,1,\dots,d \} $. Then we
    can use \ref{lm:ekr_N2} to prove that $ \mathbb{C}(\mathcal{A}) $ satisfies
    the conditions of \ref{th:ekr_commutative}. Therefore, for every matrix
    in $ \mathbb{C}(\mathcal{A}) $ there is a set $ \{ E_1^A, \dots, E_k^A \} $
    idempotent, paiwise orthogonal matrices such that $ A $ is a linear combination
    of them. Consider the set
    \[ S = \cup_{A \in \mathbb{C}(\mathcal{A})} \{ E_1^A, \dots, E_k^A \} \]
    Note that $ E_i^A $ and $ E_j^B $ are not necessarily orthogonal if
    $ A \neq B $. Consider the subset of minimal idempotent matrices in $ S $,
    $ S' $, which exists due to \ref{lm:ekr_minimal}. All elements in $ S' $
    are then pairwise orthogonal. To prove the statement we just have to show
    that $ S' $ spans $ \mathbb{C}(\mathcal{A}) $. Fix any $ F \in S $, and
    define $ F_0 $ as the sum of all $ E \in S' $ such that $ E \leq F $.
    Note that $ F_0 $ is idempotent because all matrices in $ S' $ are pairwise
    othogonal. Moreover, $ F_0 \leq F $, since
    \[ FF_0 = F(\sum_{E \in S' \mid FE = E} E) =
        \sum_{E \in S' \mid FE = E } FE =
    \sum_{E \in S' \mid FE = E } = E = F_0 \]

    Now suppose $ F_0 \neq F $. Then, it holds that
    \[ (F-F_0)^2 = F^2 - 2FF_0 + F_0^2 = F -2F_0 + F_0 = F - F_0  \]
    so $ F - F_0 $ is an idempotent matrix. Notice that
    \[ F(F-F_0) = F^2 -F_0 = F-F_0 \]
    so $ (F-F_0) \leq F $. This implies the existence of a non-zero
    minimal idempotent matrix $ G $ such that $ G \leq F-F_0 \leq F $, but
    this contradicts the definition of $ F_0 $. So $ F = F_0 $, and we have
    proved that $ \mathbb{C}(\mathcal{A}) $ is spanned by $ S' $. Note that
    the matrices in $ S $ must be linearly independent due to being orthogonal,
    so $ S' = \{ E_1, \dots, E_k \} $ is, in fact, a basis for
    $ \mathbb{C}(\mathcal{A}) $.

    We now prove the properties of these matrices $ \{ E_1, \dots E_k \} $:
    \begin{enumerate}[label=(\roman*)]
        \item Note that $ J = A_0 + A_1 + \dots + A_d \in \mathbb{C}(\mathcal{A})  $.
            Define $ E_0 := f_1(J)m_{J,1}(J) \in S $ following the notation in
            \ref{th:ekr_commutative}. We know that it is idempotent, so we
            just have to show that it has rank 1, i.e., its column space has
            dimension 1. Suppose that
            $ p(x) := p_0 + p_1 x + p_2 x^2 + \dots + p_m x^m $.
            Note that $ J^i $ has rank $ 1 $ for all $ i \in \mathbb{N} $,
            so $ p(J) $ is a matrix of rank $ 1 $ plus the matrix $ p_0 I $.
            It is clear then that the dimension of the column space over
            $ \mathbb{C} $ of $ p(J) $ is $ 1 $. From this we get that $ E_0 $
            is the product of two rank one matrices, so it has rank at most
            one.
        \item We know that $ I \in \mathbb{C}(\mathcal{A}) $, so there exist
            $ a_0, a_1, \dots, a_d \in \mathbb{C} $ such that
            \[ I = \sum_{ i=0 }^{ d } a_iE_i \]
            We just have to show that $ a_i = 0,\forall i \in \{ 0,1,\dots,d \} $.
            Note that, for any $ i $, it holds that
            \[ E_i = I E_i = (\sum_{ i=0 }^{ d } a_iE_i) E_i = a_i E_i^2
            = a_i E_i \]
            so $ a_i = 1 $.
        \item For a given $ i \in \{ 0,1, \dots, d \} $,
            $ E_i = f_i(A)m_{A,i}(A) $ for some
            $ A \in \mathbb{C}(\mathcal{A}) $. Then,
            \[ E_i^T = f_i(A)^T m_{A,i}(A)^T = f_i(A^T) m_{A,i}(A^T) \]
            Clearly $ A^T = (\sum_{ i=0 }^{ d } \alpha_i A_i )^T =
            \sum_{ i=0 }^{ d } \alpha_i A_i^T \in \mathbb{C}(\mathcal{A}) $,
            since for any $ i \in {0,1,\dots,d} $, $ A_i^T \in \mathcal{A} $.
            Therefore, $ E_i^T \in S $. Suppose $ E_i^T $ is not minimal.
            Then, there exists an $ F \in S $, $ F \neq E_i^T $, such that
            $ F \leq E_i^T \iff E_i^TF = F $. Since $ \mathbb{C}(\mathcal{A}) $
            is a commutative algebra, it then holds that
            \[ (E_i^TF)^T = F^T \iff F^T E_i = F^T \iff E_i F^T = F^T \]
            that is, $ F^T \leq E_i $, which contradicts the fact that $ E_i $
            is minimal. We conclude that $ E_i^T $ mut be minimal.
        \item It is immediate from the fact that $ \mathbb{C}(\mathcal{A}) $
            is closed under the $ Schur $ multiplication.
        \item It is immediate from the fact that any $ E_i, E_j \in S' $ are
            pairwise othogonal and idempotent.
    \end{enumerate}
\end{proof}

\begin{lemma}
    The idempotent matrices $ \{ E_0, E_1, \dots, E_d \} $ that form a basis
    of $ \mathbb{C}(\mathcal{A}) $ are also Hermitian.
\end{lemma}
\begin{proof}
    We have already shown that $ \mathbb{C}(\mathcal{A}) $ is closed under
    conjugation. Therefore, for any $ i \in \{ 0,1,\dots,d \} $, it holds that
    there exist $ a_{i0}, a_{i1}, \dots, a_{id} \in \mathbb{C} $ such that
    \[ E_i^* = \sum_{ i=0 }^{ d } a_{ij} E_j \]
    Note that, for all $ j \in \{ 0,1,\dots,d \} $, it holds that
    \[ E_i^* E_j = a_j E_j \]
    Suppose that $ E_i^* $ is not minimal. Then there is an idempotent matrix
    $ F \in \mathbb{C}(\mathcal{A}) $, $ F \neq E_i^* $, such that
    $ F \leq E_i^* \iff E_i^*F = F $. Since $ \mathbb{C}(\mathcal{A}) $ is
    commutative, this implies that
    \[ (E_i^* F)^* = F^* \iff F^* E_i = F^*  \iff E_i F^* = F^* \]
    which contradicts the fact that $ E_i $ is minimal. Therefore, $ E_i^* $ is
    minimal, and so for any $ i \in \{ 0,1,\dots,d \} $ there is a
    $ j \equiv j(i) \in \{ 0,1,\dots,d \} $ such that $ E_i^* E_j(i) \neq 0 $.

    We only have to prove that this $ j(i) $ is $ i $ itself. We know that, for
    all $ k \in \{ 0,1,\dots,d \} \setminus {j(i)} $,
    \[ E_i^* E_k = 0 \iff a_k = 0 \]
    But, since $ tr(E_i^* E_i) > 0 $, it must also hold that
    \[ E_i^* E_i \neq 0 \iff a_i \neq 0 \]
    so we conclude that $ j(i) = i $ and $ E_i^* = a_i E_i $. Using the fact that
    $ tr(E_i) = tr(E_i^*) $, we conclude that $ a_i = 1 $ and $ E_i^* = E_i $.
\end{proof}

We can use the basis of idempotents, $ \{ E_0, E_1, \dots, E_d \} $, to state
some key identities which are fundamental to the study of association schemes.

\begin{definition}
    Let $ \mathcal{A} = \{ A_0, A_1, \dots, A_d \} $ be an association scheme.
    For any given $ i $ in $ \{ 0,1,\dots,d \} $, we define the $ p_i(j) $ as
    the $ j $-th coordinate of $ A_i $ in the basis $ \{ E_0, E_1, \dots, E_d \} $,
    that is,
    \[ A_i = \sum_{ j=0 }^{ d } p_i(j)E_j \]
    The values $ p_i(j) $ are called the \textbf{eigenvalues} of $ \mathcal{A} $.
\end{definition}

The chosen name for the $ p_i(j) $ is not coincidental. Since the $ E_i $ are
idempotent and pairwise orthogonal, we have that
\[ A_iE_j = p_i(j)E_j \]
from which it is immediate that the columns of $ E_j $ are eigenvectors of
$ A_i $, with eigenvalue $ p_i(j) $. Note that $ p_i(0) $ is the valency of the
of the graph $ X_i $, that is, the number of edges that are associated to each
vertex:
\begin{proof}
    Note that
    \[ A_i E_0 = p_i(0) E_0 \iff A_i J = p_i(0) J \]
    By looking at the LHS it is clear that the $ x $-th row of the resulting
    matrix is precisely the valency of $ x $ in $ X_i $. Then, the RHS tells us that this
    valency is constant for all $ x \in X $ with value $ p_i(0) $.
\end{proof}
\begin{remark}
    Note that, since we are working with commutative, symmetric association
    schemes, the $ A_i $ matrices are symmetric matrices over $ \mathbb{R} $,
    and so their eigenvalues are also in $ \mathbb{R} $.
\end{remark}
The valency is usally referred to as $ v_i := p_i(0) $.

We can make analog constructions using $ \{ A_0, A_1, \dots, A_d \} $ as the
basis for $ \mathbb{C}(\mathcal{A}) $:

\begin{definition}
    Let $ \mathcal{A} = \{ A_0, A_1, \dots, A_d \} $ be an association scheme.
    For any given $ j \in \{ 0,1,\dots,d \} $, we define the $ q_j(i) $ as $ n $
    times the $ i $-th coordinate of $ E_j $ in the basis
    $ \{ A_0, A_1, \dots, A_d \} $, that is,
    \[ E_j = \frac{1}{n} \sum_{ i=0 }^{ d } q_j(i)A_i \]
    The scalars $ q_j(i) $ are called the dual eigenvalues of $ \mathcal{A} $.
\end{definition}

\begin{lemma}
    Let $ \mathcal{A} = \{ A_0,\dots,A_d \} $ be an association scheme. Then,
    for any $ j \in \{ 0,1,\dots,d \} $, it holds that
    \[ q_j(0) = \text{tr}(E_j) = \text{rk}(E_j) \]
\end{lemma}
\begin{proof}
    We start by noting that complex idempotent matrices can only have eigenvalues $ 0 $
    or $ 1 $. Then, by changing to the eigenvector basis, it is easy to see that
    the trace of an idempotent matrix is precisely its rank.

    By the definition of $ q_j(i) $, it holds that
    \[ \text{tr}(E_j) = \frac{1}{n}\sum_{ i=0 }^{ d }q_j(i)\text{tr}(A_i) \]
    Since $ tr(A_i) = 0 $ if $ i\neq 0 $ and $ A_0 = I $, it is clear that the
    right hand side is just $ q_j(0) $.
\end{proof}

The values $ q_j(0) $, for $ j \in \{ 0,1,\dots,d \} $, is often referred to as
the \textbf{multiplicities} of the association scheme, denoted by $ m_j $.

\begin{proposition}
    \label{prop:association_schemes_pq_identity}
    Let $ \mathcal{A} = \{ A_0, A_1, \dots, A_d \} $ be an association scheme,
    let $ \{ E_0, E_1, \dots, E_d \} $ its orthogonal idempotent matrix basis.
    Then, for any given $ i,j \in \{ 0,1,\dots,d \} $ it holds that
    \[ \sum_{ k=0 }^{ d } p_k(i) q_j(k) =
        \begin{cases}
            0, i \neq j \\
            n, i = j
        \end{cases}
    \]
\end{proposition}
\begin{proof}
    Fix some $ i,j \in \{ 0,1,\dots,d \} $. Looking at the definitions of
    $ q_j(i) $ and $ p_i(j) $, we get that
    \[
        E_j = \frac{1}{n} \sum_{ k=0 }^{ d } q_j(k) A_k \iff
        E_j = \frac{1}{n} \sum_{ k=0 }^{ d } q_j(k) \sum_{ l=0 }^{ d } p_k(l) E_l
    \]
    from which
    \[
        E_j E_i = \frac{1}{n} \sum_{ k=0 }^{ d } q_j(k) \sum_{ l=0 }^{ d }
            p_k(l) E_l E_i
        \iff
        n\delta_{ij} = \sum_{ k=0 }^{ d } q_j(k) p_k(i)
    \]
\end{proof}

\begin{definition}
    We define the \textbf{eigenmatrix} of the association scheme as the matrix
    $ P $ whose coefficients are given by
    \[ P_{i,j} = p_j(i) \]
    We define the \textbf{dual eigenmatrix} of the association scheme as the
    matrix $ Q $ whose coefficients are given by
    \[ Q_{i,j} = q_j(i) \]
\end{definition}

Note that, from \ref{prop:association_schemes_pq_identity}, it is immediate that
\begin{equation}
    PQ = nI
\end{equation}
Similarly, it holds that
\begin{equation}
    QP = nI
    \label{eq:QP_identity}
\end{equation}

Note for a fixed $ A_i $, the column space of each of
the $ E_j $ defines an eigenspace of $ A_i $,
with all of the column spaces beign pair-wise disjoint due to the $ E_j $ being
orthogonal. What we have not seen yet is that the $ E_j $ define \textit{all}
of the eigenspaces of $ A_i $. Since the sum of the dimensions of the complex
eigenspaces of any complex $ n\times n $ matrix must be equal to $ n $, it is
sufficient to show that the sum of all the multiplicities is equal to $ |X| $.

\begin{proposition}
    Let $ m_0, m_1, \dots, m_d $ be the multiplicities of an associtiation
    scheme over the set $ X $. Then, it holds that
    \[ \sum_{ k=0 }^{ d } m_k = |X| \]
\end{proposition}
\begin{proof}
    From \eqref{eq:QP_identity} we get that
    \[ |X| = (QP)_{00} = \sum_{ k=0 }^{ d } Q_{0k} P_{k0} =
    \sum_{ k=0 }^{ d } q_k(0) p_0(k)   \]
    Using the fact that $ p_0(k) = 1 $, we get that
    \[ |X| = \sum_{ k=0 }^{ d } m_k \]
\end{proof}

Therefore, we may conclude the following:
\begin{corollary}
    For any given $ i \in \{ 0,1,\dots,d \} $, the column spaces of the matrices
    $ \{ E_0, E_1, \dots, E_d \} $ are the eigenspaces of the matrix $ A_i $.
\end{corollary}

One can define an inner product in $ \mathbb{C}(\mathcal{A}) $. If $ M,N \in
\mathbb{C}(\mathcal{A}) $, we define a bilinear form $ \langle M,N \rangle $ as
\[ \langle M,N \rangle := tr(M^*N) \]
This satisfies the properties necessary for it to be an inner product:
\begin{proof}
    \begin{itemize}
        \item $ \langle N,M \rangle = tr(N^*M) = tr(MN^*) = \overline{tr(M^*N)} $.
        \item $ \langle M,a N_1 + b N_2 \rangle = tr(M^*(aN_1 + bN_2)) =
            tr(M^*(aN_1) + M^*(bN_2)) = atr(M^*N_1) + btr(M^*N_2) =
            a\langle MN_1 \rangle b \langle MN_2 \rangle $.
        \item $ \langle M,M \rangle = tr(M^*M) = \sum_{ i=0 }^{ n }
            \overline{M_{ii}}M_{ii} = \sum_{ i=0 }^{ n } \mid M_{ii} \mid^2
            > 0 $.
    \end{itemize}
\end{proof}

We can define this inner product in an equivalent way that is sometimes more
convenient. First, for a matrix $ A \in \mathbb{C}(\mathcal{A}) $, define $ sum(A) $
as the sum of all the entries in $ A $. Then, we can rewrite
\[ \langle M,N \rangle = sum(\overline{M}N) \]

Note that both $ \{ E_0, E_1, \dots, E_d \} $ and $ \{ A_0, A_1, \dots, A_k \} $
form an orthogonal basis with respect to this inner product.

It is particularly interesting to consider the inner products between $ A_i $
and $ E_j $ for all $ i,j \in \{ 0,1,\dots,d \} $. Following the first
definition, we get that
\[ \langle A_i,E_j \rangle = tr(A_i^TE_j) = \overline{tr(E_j^* A_i)} =
\overline{tr(E_jA_i)} = \overline{tr(A_iE_j)} = tr(\overline{p_i(j)E_j}) =
\overline{p_i(j)}m_j\]
while with the second definition we get that
\[ \langle A_i,E_j \rangle = sum(E_j \circ A_i) =
\sum_{ k=0 }^{ d } \frac{1}{n} q_j(i) sum(A_k \circ A_i) = q_j(i)v_i \]

Thus, we have that
\[ \frac{\overline{p_i(j)}}{v_i} = \frac{q_j(i)}{m_j} \]

We will use $ \Delta_v $ and $ \Delta_m $ to denote the diagonal matrix of
valencies and multiplicities, respectively. Then, we can express the previous
equality as
\[ \Delta_m P = Q^* \Delta_v \]

As we will see in the next section, the action of a group $ G $ on the set $ X $
over which we have an association scheme can have a lot of interesting properties.
For this reason, for any given $ g \in G $ we will define $ P_g $ as the
permutation matrix of $ g $ over $ X $, that is, the matrix indexed by $ X $
whose $ (x,y-x) $ element is
\[
    (P_g)_{xy} =
    \begin{cases}
        1 &\text{ if y = gx,} \\
        0 &\text{ otherwise}
    \end{cases}
\]

This allows us to define the following:
\begin{definition}
    The \textbf{centralizer algebra}, or \textbf{commutant}, of an association scheme
    $ (X,\mathcal{R}) $ is the algebra of all complex matrices that commute with
    $ P_g $ for all $ g \in G $. We denote it as $ \mathcal{C}(X) $.
\end{definition}

In \parencite[Lemma 3.2.1]{godsil_erdoskorado_2015}, a fundamental property of the
centralizer algebra of association schemes is stated, but not proven. A proof is
provided below.

\begin{lemma}
    The commutant of a set of $ n \times n $ permutation matrices is a Schur-closed
    matrix algebra that contains $ I $ and $ J $ and is closed under transposition.
\end{lemma}
\begin{proof}
    The matrix $ I $ is precisely the permutation matrix of the identity permutation,
    and is clearly commutative.

    For any $ g \in G $, the $ x $-th row of the matrix $ PJ $ is an array whose
    every entry is the number of elements $ y \in X $ such that $ x = gy $. Note
    that, for every such $ y $, there is a $ z := gy \in X $ such that
    $ z = gx $, and that $ JP $ is a matrix whose $ x $-th column is an array
    where all the entries are the number of elements $ z \in X $ such that
    $ z = gx $. Therefore, it must hold that $ JP = PJ $.

    Supose now that $ A,B $ are two matrices in the commutant, and consider any
    $ \lambda \in \mathbb{C} $ and $ g \in G $. Then it holds that
    \begin{equation*}
        \begin{split}
            & (A+B) P_g = AP_g + BP_g = P_gA + P_gB = P_g(A+B) \\
            &(AB) P_g = A(BP_g) = (AP_g)B = P_g(AB) \\
            &(\lambda A)P_g = \lambda(AP_g) = \lambda (P_gA) = P_g(\lambda A)
        \end{split}
    \end{equation*}
    which proves that the commutant is a matrix algebra. Furthermore,
    \[ A^TP_g = (P_g^TA)^T = (P_{g^{-1}}A)^T = (AP_{g^{-1}})^T = P_g A^T \]
    so it is closed under transposition.

    We only have left to check wether it is closed under the Schur multiplication.
    Let $ P $ be any permutation matrix of the form
    \[
        P = \left (
        \begin{array}{ccc}
            \horzbar & e_{r_1}^T & \horzbar \\
                     & \vdots & \\
            \horzbar & e_{r_n}^T & \horzbar
        \end{array}
        \right ), \quad
        P = \left (
        \begin{array}{ccc}
            \vline  &  & \vline \\
            e_{c_1} & \cdots & e_{c_n} \\
            \vline & & \vline
        \end{array}
        \right )
    \]
    for some permutations $ (c_1,\dots,c_n) $, $ (r_1,\dots,r_n) $ of
    $ (1,\dots,n) $, and where $ e_i $ is the
    $ i $-th canonical column vector. Then, $ A $ belongs to the commutant if
    and only if, for all $ i,j $,
    \[ (PA)_{ij} = (AP)_{ij} \iff a_{r_i,j} = a_{i,c_j}  \]
    Therefore, for any $ A,B $ in the commutant, it holds that
    \[ (A \circ B)_{r_i,j} = a_{r_i,j}b_{r_i,j} = a_{i,c_j}b_{i,c_j} =
    (A \circ B)_{i,c_j} \]
    and so $ A \circ B $ is also in the commutant, proving that the commutant
    is closed under the Schur multiplication.
\end{proof}

Another concept that is often talked about in the literature, and that we will
utilize at the end of this section, is the inner and outer distributions.
\begin{definition}
    Let $ (X,\mathcal{R}) $ be $ d $-class association scheme, and let $ Y $
    be a nonempty subset of $ X $. We define the \textbf{inner distribution} of
    $ Y $ as the row vector $ a = ( a_0, a_1, \dots,a_d ) $ where
    \[ a_i = \frac{1}{|Y|}|(Y\times Y)\cap R_i| \]
\end{definition}
\begin{remark}
    The inner distribution is sometimes found in the literature as
    \[ a_i = \frac{1}{|Y|}\boldsymbol{x_Y}^T A_i \boldsymbol{x_Y} \]
    where $ \boldsymbol{x_Y} $ is the characteristic vector of $ Y $, that is,
    \[ \boldsymbol{x_Y}(t) = \begin{cases}
        &1 \text{ if } t \in Y \\
        &0 \text{ otherwise}
    \end{cases} \]
    One can easily check that this two definitions are equivalent.
\end{remark}

\begin{lemma}[{\parencite[Theorem 30.3]{van_lint_course_2001}}]
    The distribution vector $ a $ of a nonempty subset $ Y $ of an association
    scheme $ (X,\mathcal{R}) $ satisfies
    \[ aQ \geq 0 \]
    where $ 0 $ is the $ (d+1) $-dimensional zero row vector.
\end{lemma}
\begin{proof}
    Since $ E_j $ is idempotent and symmetric, it holds that
    \begin{equation*}
        \begin{split}
        0 \leq |\boldsymbol{x_Y}^TE_j|^2 = \boldsymbol{x_Y}^T E_j^T E_j \boldsymbol{x_Y} = \boldsymbol{x_Y}^T E_j \boldsymbol{x_Y}
        =& \frac{1}{|X|}\boldsymbol{x_Y}^T(\sum_{ i=0 }^{ d }Q_{ij}A_i)\boldsymbol{x_Y} = \\
        =& \frac{|Y|}{|X|}(\sum_{ i=0 }^{ d }Q_{ij}\boldsymbol{x_Y}^TA_i\boldsymbol{x_Y})
        \end{split}
    \end{equation*}
\end{proof}

The lemma above, when applied to the association scheme obtained over matrices,
gives us an analog of the \textit{MacWilliams inequalities} established for
classical codes.

\begin{definition}
    Let $ (X,\mathcal{R}) $ be $ d $-class association scheme, and let $ Y $
    be a nonempty subset of $ X $. We define the outer distribution of $ Y $
    as the $ |X|\times(d+1) $ matrix B where
    \[ B_{xi} := |\{ y \in Y \mid (x,y) \in R_i \} | = (A_i\chi)_x \]
\end{definition}

\begin{lemma}[{\parencite[Lemma 2.5.1]{brouwer_distance-regular_1989}}]
    The inner distribution $ a $ and outer distribution $ B $ of a nonempty
    subset $ Y $ of an association scheme $ (X,\mathcal{R}) $ satisfy
    \[ (BQ)_{xj} = n\chi^TE_j\chi, \quad (|Y|aQ)_j = |X|\chi_y^TE_j\chi \]
    \label{lm:in_out_dist}
\end{lemma}
\begin{proof}
    \begin{equation*}
        (BQ)_{xj} = \sum_{ i } B_{xi}Q_{ij} = \sum_{ i } (A_i\chi)_xQ_{ij} =
        |X|(E_j\chi)_x
    \end{equation*}
    \begin{equation*}
        (|Y|aQ)_j = (\chi^TBQ)_j = |X|\chi^TE_j\chi
    \end{equation*}
\end{proof}

\section{Translation association schemes}
In this section we will over some properties which are specific to a certain type
of association schemes called translation association schemes.

\begin{definition}[{\parencite[Section 2.10]{brouwer_distance-regular_1989}}]
    A \textbf{translation association scheme} is an association scheme
    $ (X, \mathcal{R}) $ in which the underlying set $ X $ has the structure of
    an abelian group and, for all classes $ R \in \mathcal{R} $, it holds that
    \[ (x,y) \in R \iff (x+z,y+z) \in R, \forall z \in X \]
\end{definition}

It is clear that the association scheme obtained with the rank distance over the
set of matrices with coefficients in $\mathbb{F}_{q}$ must also be a
translation association scheme.

\begin{comment}

To give a complete proof, we will use the fact that finite group actions over
finite abelian groups give constructions of translation association schemes:

\begin{theorem}[{\parencite[Theorem 2]{kim_duality_2011}}]
    \label{th:tas_construction}
    Suppose $ G $ is a finite group acting on an abelian group $ X $ with
    $ \mathcal{O}_0, \mathcal{O}_1, \dots, \mathcal{O}_d $ being the orbits of
    this action. Suppose that this action has the following properties:
    \begin{enumerate}[label=(\alph*)]
        \item \label{th:tas_construction:item:O0} $
            \mathcal{O}_0 = \{ 0 \} $;
        \item \label{th:tas_construction:item:distributive}
            $ g(x + x') = gx + gx' $, $ \forall g \in G $ $ \forall x,x' \in X $;
        \item \label{th:tas_construction:item:minus}
            $ x \in \mathcal{O}_i \Rightarrow -x \in \mathcal{O}_i, \forall
            i \in \{ 0,1,\dots,d \} $.
    \end{enumerate}
    Then, $\mathfrak{X}_G = (X, \mathcal{R} = \{ R_0, R_1, \dots, R_d \} $, where
    the relation is given by
    \[ (x,y) \in R_i \iff y-x \in \mathcal{O}_i \]
    is a symmetric translation scheme.
\end{theorem}
\begin{proof}
    We will start by proving that all axioms in
    \ref{def:sym_association_scheme_partition}:
    \begin{enumerate}[label=(\roman*)]
        \item $ (x,y) \in X \times X \Rightarrow y-x \in \mathcal{O}_i $ for a
            unique $ i \in \{ 0,1,\dots,d \} $ because the orbits of the action
            of $ G $ over $ X $ partition $ X $. Therefore, every $ (x,y) \in
            X \times X $ lies in $ R_i $ for a unique $ i \in \{ 0,1,\dots,d \} $,
            and so $ \mathcal{R} $ partitions $ X \times X $.
        \item $ \forall x \in X, x-x = 0 \in \mathcal{O}_0 \Rightarrow (x,x) \in R_0$.
        \item Given a fixed $ i \in \{ 0,1,\dots,d \} $, due to
            \ref{th:tas_construction:item:minus} it holds that
            \[ (x,y) \in R_i \iff y-x \in \mathcal{O}_i \iff
            x-y \in \mathcal{O}_i \iff (y,x) \in \mathcal{0}_i \]
        \item Fix $ i,j,k \in \{ 0,1,\dots,d \} $. We want to prove that, for
            any $ x,y,x',y' \in X $ such that $ u := y-x \in \mathcal{O}_k $ and
            $ v := y'-x' \in \mathcal{O}_k $, it holds that:
            \[ \mid T_{k,i,j}(x,y) :=
                \{ z \in X \mid z-x \in \mathcal{O}_i, y-z \in \mathcal{O}_j \}
                \mid = \mid
                T_{k,i,j}(x',y') :=
                \{ z \in X \mid z-x' \in \mathcal{O}_i, y'-z \in \mathcal{O}_j \}
                \mid
            \]
            We start by constructing a bijection
            \begin{equation*}
            \begin{split}
                \Phi_{x,y}: \{ z \in X \mid z-x \in \mathcal{O}_i, y-z \in \mathcal{O}_j \}
                \rightarrow &
                \{ t \in X \mid u-t \in \mathcal{O}_i, t \in \mathcal{O}_j \} \\
                z \rightarrow & \Phi_{x,y}(z) := y - z
            \end{split}
            \end{equation*}
            The codomain is well defined, since for any $ z $ in the LHS,
            $ \Phi_{x,y}(z) $ is such that $ u - \Phi_{x,y}(z) = (y-x) - (y-z) = z-x \in
            \mathcal{O}_i $ and $ \Phi_{x,y}(z) = y-z \in \mathcal{O}_j $.

            Suppose $ z_1 \neq z_2 $, then $ \Phi_{x,y}(z_1) - \Phi_{x,y}(z_2) =
            (y-z_1) - (y-z_2) = z_2 - z_1 \neq 0 $. Therefore, $ \Phi_{x,y} $ is
            injective.

            Suppose $ t \in X $ is such that $ u-i \in \mathcal{O}_i $ and
            $ t \in \mathcal{O}_j $. Then we can define a $ z := y-t \in X $
            such that $ \Phi_{x,y}(z) = y - (y-t) = t $ and $ z-x = y-t-x = (y-x) - t
            = u - t \in \mathcal{O}_i $, $ y-z = y-(y-t) = t \in \mathcal{O}_j $.
            Therefore, $ \Phi_{x,y} $ is exhaustive.

            $ \Phi_{x,y} $ is a bijection, and so is the analog map for the points
            $ x',y' $, $ \Phi_{x',y'} $.

            Note that $ G $ acts transitively on each of its orbits. Therefore,
            there exists an $ h
            \in G $ such that $ v = hu $. We will use this to construct another
            two bijections
            \begin{equation*}
            \begin{split}
                \Psi: \{ z \in X \mid u-z \in \mathcal{O}_i, z \in \mathcal{O}_j \}
                \rightarrow
                &\{ t \in X \mid v-t \in \mathcal{O}_i, t \in \mathcal{O}_j \} \\
                z \rightarrow &\Psi(z) := hz
            \end{split}
            \end{equation*}

            The codomain is well defined, since due to
            \ref{th:tas_construction:item:distributive}
            for any $ z $ in the LHS it holds that
            $ v - \Psi(z) = hu - hz = h(u-z) \in \mathcal{O}_i \iff u-z \in
            \mathcal{O}_i$ and
            $ hz \in \mathcal{O}_j \iff z \in \mathcal{O}_j $.

            Suppose that $ z_1 \neq z_2 $. Note that $ hz = 0 \Rightarrow
            h^{-1}(hz) = 0 \iff (h^{-1}h)(z) = h^{-1}(0) \iff z = 0 $. Then,
            $ \Psi(z_1) - \Psi(z_2) = hz_1 - hz_2 = h(z_1-z_2) \neq 0 $, and
            $ \Psi $ is injective.

            Suppose $ t \in X $ is such that $ v - t \in \mathcal{O}_i $ and
            $ t \in \mathcal{O}_j $. Then we can define $ z := h^{-1}t $, such
            that $ \Psi(z) = h(h^{-1})(t) = (hh^{-1})(t) = t $ and
            $ u-z = h^{-1}v - h^{-1}t = h^{-1}(v-t) \in \mathcal{O}_i \iff
            \mathcal{O}_i $ and $ z = h^{-1}t \in \mathcal{O}_j \iff
            t \in \mathcal{O}_j $.

            Therefore, we have shown that we can go from $ T_{k,i,j}(x,y) $ to
            $ T_{k,i,j}(x',y') $ with the bijection
            \[ \Phi_{x',y'}^{-1} \circ \Psi \circ \Phi_{x,y} \]
            and it is clear that both sets have the same cardinality.
    \end{enumerate}
\end{proof}

Since matrices form an abelian group under matrix addition, we just have to find
a group acting on the matrices such that its orbits are precisely
\[ \mathcal{O}_i := \{ A \in M_n(\mathbb{F}) \mid rank(A) = i \} \]
and it satisfies the hypothesis of $ \ref{th:tas_construction} $.

We will consider $ G = GL(n,q) $, that is, the set of non-singular $ n \times
n $ matrices over $ M_n(\mathbb{F}_{q}) $. First we check that $ G $ satisfies the
previous hypothesis:
\begin{enumerate}[label=(\alph*)]
    \item For any $ S \in GL(n,q) $, it is clear that $ S0 = 0 $.
    \item For any $ S \in GL(n,q) $ and any matrices $ A,B $, it holds
        that:
        \[ S(A+B) = SA + SB \]
    \item For any matrix $ A $, it is immediate that $ rank(A) = rank(-A) $.
\end{enumerate}
Now we will show that the orbits of the action of $ G $ over an additive group
of matrices in $ M_n(\mathbb{F}_{q}) $ do in fact partition the matrices
according to their rank. Suppose two matrices $ A,B $ have rank $ r $. Then,
there are $ r $ columns in both $ A $ and $ B $ that are linearly independent.
Let $ (i_1, \dots, i_r) $ and $ (j_1, \dots, j_r) $ be the indexes of these
columns of A and B, respectively. There is a basis change matrix, $ M_A $, such
that $ a_{i_k} = M_A e_{i_k} $, where $ a_{i_k} $ is the corresponding column of
$ A $ and $ e_{i_k} $ is the corresponding cannonical basis vector. There is an
analog matrix $ M_B $ such that $ e_{j_k} = M_B b_{j_k} $. Let $ P $ be the
permutation matrix that brings the columns in $ (j_1, \dots, j_r) $ to the
columns in $ (i_1, \dots, i_r ) $. Then, we have that
\[ A = (M_A^{-1} P M_B) B \]
where $ S := M_A^{-1} P M_B \in GL(n,q) $.

\end{comment}

Working with a translation association scheme is useful because we can use the
structure of the abelian group $ X $ to give another meaning to some of the
identities that hold for general association schemes. For this, an inner product
on $ X $
\footnote{This must not be confused with the inner product defined previously
over the matrices in $ \mathbb{C}(\mathcal{A}) $. Even when $ X $ is a set of
squared matrices, the two are not related.}.
is necessary. Note that, since $ X $ is an abelian finite group, the existence
of an inner product is guaranteed: the fundamental theorem of finite abelian
groups states that $ X $ is a direct sum of cyclic subgroups of prime-power order,
that is,
\[
    X = ( \mathbb{Z} / e_1\mathbb{Z} ) \oplus ( \mathbb{Z} / e_2\mathbb{Z} )
    \oplus \dots \oplus ( \mathbb{Z} / e_m\mathbb{Z} )
\]
and if we denote the lowest common multiple of $ e_1, \dots, e_m $ by $ e $, then we
can define the inner product of $ X $ as
\[ \langle x,y \rangle = \sum_{ i=1 }^{ m } \frac{e}{e_i} x_i y_i \]
where $ x, y \in X $, $ x = (x_1,\dots,x_m), y = (y_1, \dots, y_m) $. Sometimes
it might be useful to use a different inner product, if it exists.

We will now reintroduce the concept of the centralizer algebra. When dealing with
translation association schemes, the set $ X $ is now isomorphic to its group
of permutations. Therefore, the permutation matrices can be expressed as $ P_x $,
where $ x \in X $, and for every $ a,b \in X $
\[ (P_x)_{ab} =
    \begin{cases}
        & 1 \text{ if b = a + x} \\
        & 0 \text{ otherwise}
    \end{cases}
\]

It is easy to check that the properties hold for the permutation matrices of
a finite abelian set:
\begin{equation*}
      P_x P_y = P_{x+y},\;
      P_x P_y = P_y P_x,\;
      (P_x)^s = P_{sx},\;
      P_x \circ P_y = \delta_{xy}P_x,\;
      P_x^T = P_{-x}
\end{equation*}

\begin{definition}
    The \textbf{restricted centralizer algebra} of a translation association scheme
    $ (X,\mathcal{R}) $, $ \mathcal{C}_r(X) $, is the real subalgebra of real,
    symmetric matrices in $ \mathcal{C}(X) $.
    It is spanned by $ P_x + P_{-x} $, for all $ x \in X $.
\end{definition}

This can analogously be defined for regular association schemes.

\begin{lemma}
    Every adjacency matrix $ A_i $ of a translation association scheme on $ X $
    can be expressed as
    \[ A_i = \sum_{ x\in N_i } P_x  \]
    where
    \begin{equation}
        N_i = \{ x \in X \mid (0,x) \in R_i \}
    \end{equation}
    \label{lm:Ai_as_sum_of_perm}
\end{lemma}
\begin{proof}
    In a translation associan scheme, for any $ a,b \in X $, it holds that
    \[ (a,b) \in N_i \iff b-a \in N_i \]
    Note that, for every $ z \in N_i $,
    \[ (P_z)_{ab}
        \begin{cases}
            & 1 \text{ if } b-a = z \\
            & 0 \text{ otherwise}
        \end{cases}
    \]
Also, we observe that $ P_x \circ P_y = \delta_{xy} P_x $, that is, for two
different permutation matrices, there are no entries which are one in both.
Therefore,
    \[
        (\sum_{ x \in N_i } P_x)_{ab} =
        \begin{cases}
            & 1 \text{ if } b-a = x \text{ for some } x \in N_i \\
            & 0 \text{ otherwise }
        \end{cases}
    \]
which proves our statement.
\end{proof}

From this, the following is immediately clear:
\begin{lemma}
    The Bose-Mesner algebra of a translation
    association scheme, $ \mathbb{C}(\mathcal{A}) $, is a subalgebra of
    $ \mathcal{C}_r(X) $.
    \label{lm:bm_subalg}
\end{lemma}

\begin{definition}
    We define the \textbf{character group} $ X^* $ of a finite abelian group
    $ X $ as the group of all functions $ \chi : X \rightarrow S^1 $ such that
    \[ \chi(x+y) = \chi(x) \chi(y), \; \forall x,y \in X \]
    where the group operation is defined as
    \[ \chi_1\chi_2(x) = \chi_1(x)\chi_2(x) \]
\end{definition}
\begin{remark}
    Characters are often defined as functions from $ X $ to $ \mathbb{C}^* $,
    but note that for every $ x \in X $ there is an $ m_x \in \mathbb{N} $ such
    that $ m_xx = 0 $, and therefore
    \[
        1 = \chi(0) = \chi(m_xx) = \chi(x)^{m_x} \implies
        |\chi(x)| = 1
    \]
    for all $ \chi \in X^* $.
\end{remark}

For each character in $ \chi $, we can construct an $ |X| $-dimensional
vector indexed by $ x \in X $ whose $ x $-th coordinate is $ \chi(x) $,
that is,
\begin{equation}
    \chi_x = \chi(x)
\end{equation}

\begin{proposition}[Proposition 2.10.7, \cite{brouwer_distance-regular_1989}]
    The character group $ X^* $ of a finite abelian group $ X $ is isomorphic
    to $ X $. Moreover, there exists an isomorphism $ f : X \rightarrow X^* $
    such that $ f(x)(y) = f(y)(x) $ for all $ x,y \in X $.
    \label{prop:char_isomorph}
\end{proposition}
\begin{proof}
    Note that $ X $ can be expressed a direct product of cyclic groups due to
    it being finite and abelian: $ X = X_1 \times \dots \times X_r $. Let
    $ \{ x_1, \dots, x_r \} $ be a basis of $ X $, and let $ m_i $ be the order
    of $ x_i $, for $ i \in \{ 0,\dots,r\} $. Then, we may define a set of
    characters, $ \{ \xi_1, \dots, \xi_r \} $, such that $ \xi_i(x_j) = 1 $ for
    all $ j \neq i $ and $ \xi_i (x_j) = e^{\frac{2\pi i}{m_i}} $. We will show
    that this set acts as a basis of $ X^* $. Let $ \chi \in X^* $. For
    any $ x \in X $, there are $ \lambda_1, \dots, \lambda_r \in \mathbb{Z} $
    such that $ x = \lambda_1 x_1 + \dots + \lambda_r x_r $, and so
    \[ \chi(x) = \chi(\lambda_1 x_1 + \dots +\lambda_rx_r) =
    \chi(x_1)^{\lambda_1} \cdots  \chi(x_r)^{\lambda_r} \]
    Note that, for each $ i \in \{ 1,\dots,r \} $ it must hold that
    \[ \chi(x_i)^{m_i} = \chi(m_ix_i) = \chi(0) = 1 \]
    Therefore, for each $ i \in \{ 1,\dots,r \} $ there is an integer $ k_i $
    such that $ \chi(x_i) = e^{\frac{2\pi i k_i}{m_i}} = \xi_i(x_i)^{k_i} $,
    and so
    \[ \chi = \xi_1^{\lambda_1 k_1} \cdots \xi_r^{\lambda_r k_r} \]
    proving that $ \{ \xi_1, \dots, \xi_r \} $ is a basis of $ X^* $.

    Let us now construct the isomophism $ f $ by mapping one basis to the other,
    that is,
    \[ f(x_i) = \xi_i \]
    Then, for any $ x,y \in X $, $ x = \lambda_1 x_1 + \dots + \lambda_r x_r $
    and $ y = \nu_1 y_1 + \dots + \nu_r y_r $
    \begin{equation*}
        \begin{split}
        f(x)(y) =
        (\xi_1^{\lambda_1} \cdots \xi_r^{\lambda_r}) (\xi_1\nu_1 + \dots + \xi_r\nu_r)
        &= \xi_1^{\lambda_1} (\nu_1x_1 + \dots + \nu_rx_r) \cdots
        \xi_r^{\lambda_r} (\nu_1x_1 + \dots + \nu_rx_r) = \\
        &= (\xi_1^{\lambda_1})^{\nu_1} \cdots (\xi_r^{\lambda_r})^{\nu_r}
        = (\xi_1^{\nu_1})^{\lambda_1} \cdots (\xi_r^{\nu_r})^{\lambda_r} = \\
        &= \xi_1^{\nu_1} (\lambda_1x_1 + \dots + \lambda_rx_r) \cdots
        \xi_r^{\nu_r} (\lambda_1x_1 + \dots + \lambda_rx_r) = \\
        &= f(y)(x)
        \end{split}
    \end{equation*}
\end{proof}

Note that each $ x \in X $ defines a character $ x^{**} $ on $ X^* $ by
$ x^{**}(\chi) = \chi(x) $. Furthermore, by \ref{prop:char_isomorph}, we have
that $ |X| = |X^*| = |X^{**}| $. Therefore, we can think of $ X^{**} $ as being
\textit{cannonically isomorphic} to $ X $, and we will use the two terms
interchangeably.

Given an inner product $ \langle \cdot,\cdot \rangle $ over $ X\times X $ and a character
$ \chi $ on $ X $, we define a \textbf{multiplicative inner product} over
$ X \times X $ as
\[ A\cdot B = \chi \langle A,B \rangle  \]
where it is clear that
\[
    A \cdot B = B \cdot A; \quad A \cdot B = 1 \iff A = 0 \text{ or } B = 0;
    \quad A \cdot (B_1 + B_2) = (A \cdot B_1)(A \cdot B_2)
\]
The character group is sometimes given a diferent definition in the literature.
In \cite{kim_duality_2011}, given any multiplicative inner product $ \cdot $,
the $ X^* $ is defined as the set of $ \chi:X \rightarrow S^1 $ such that
\[ \chi(y) = y \cdot x \]
for some $ x \in X $.
Note that every such $ \chi $ is clearly a character under our previous definition.
From \ref{prop:char_isomorph}, we deduce that this set is not only well defined
(that is, it does not depend on the choice of the multiplicative inner product),
but it also must be exactly the same as the previouly defined $ X^* $, since one
is included in the other and they have the same cardinality, $ |X| $.


\begin{proposition}[Proposition 2.10.8, \cite{brouwer_distance-regular_1989}]
    Let $ Y $ be a subgroup of $ |X| $. Then, for each $ \chi \in X^* $,
    we have that
    \begin{equation*}
        \sum_{ y \in Y } \chi (y) =
        \begin{cases}
            &| Y | \text{ if } \chi(y) = 1 \text{ for all } y \in Y, \\
            &0 \text{ otherwise}
        \end{cases}
    \end{equation*}
    \label{prop:sum_char_tas}
\end{proposition}
\begin{proof}
    Suppose there is a $ y_0 \in Y $ such that $ \chi(y_0) \neq 1 $. Then
    \[ \chi(y_0) \sum_{ y\in Y } \chi (y) =
        \sum_{ y\in Y } \chi (y+y_0) =
        \sum_{ y\in Y } \chi (y)
    \]
    which implies that $ \sum_{ y\in Y } \chi (y) = 0 $.
\end{proof}

This proposition allows us to see further into the relationship between the
translation association scheme over $ X $ and $ X^* $.

\begin{proposition}
    Let $ (X, \mathcal{R}) $ be a $ d $-class translation association scheme. For
    every association matrix $ A_j $, each character $ \chi \in X^* $ is
    an eigenvector of $ A_j $, with eigenvalue
    $ \sum_{ x \in N_j } \chi(x) $. Furthermore, iterating over all
    $ \chi \in X^* $ we get all the eigenvalues of $ A_j $.
    \label{prop:character_eigen}
\end{proposition}
\begin{proof}
    Fix any $ \chi \in X^*  $. Since $ X $ is a translation is an association
    scheme, we have that $ (x,y) \in R_j \iff y-x \in N_j $. Also, we have that
    \[ \chi(y) = \chi(y-x+x) = \chi(y-x)\chi(x) \]
    We now look at the value of the $ x $-th element of the vector $ A_j \chi $:
    \[
        (A_j\chi)_x = \sum_{ (x,y) \in R_j } \chi(y) = \sum_{ z \in N_j }
        \chi(z)\chi(x) = \left ( \sum_{ z \in X_j } \chi(z) \right ) \chi(x)
    \]
    It is therefore clear that for any $ j $, any $ \chi \in X^* $ is an
    eigenvector of $ A_j $, with the desired eigenvalue.

    Note that, any distinct $ \chi_1, \dots, \chi_k $ in $ X^* $ are linearly
    independent when taken as vectors indexed by $ X $. This can be proved by
    induction. Therefore, iterating over $ \chi \in X^* $ must give all the
    eigenvalues of $ A_j $.
\end{proof}

\begin{definition}
    The \textbf{dual scheme} $ (X^*, \mathcal{R}^*) $ of a translation association
    scheme $ (X,\mathcal{R}) $
    is the character group $ X^* $ of $ X $ with
    $ \mathcal{R}^* = \{ R_0^*, R_1^*, \dots, R_d^* \} $, where
    \[ R_i^* = \{ (\chi, \psi) \in X^*\times X^* \mid
    E_i (\chi^{-1}\psi) = \chi^{-1}\psi \} \]
    We will write $ N_i^* = \{ \eta \in X^* \mid (0,\eta) \in X\times X \} $.
\end{definition}

\begin{remark}
    For any given $ i,j \in \{ 0,1,\dots,d \} $, for all
    $ \eta \in N_i^* $ it holds that
    \begin{equation}
        P_{ij} = \sum_{ x \in N_j } \eta(x)
        \label{eq:pval_char_sum}
    \end{equation}
\end{remark}
\begin{proof}
    Directly from \ref{prop:character_eigen}.
\end{proof}

One of the most important properties of translation association schemes, as we
will see in this section, it that their corresponding dual schemes are also
translation association schemes.

\begin{definition}
    We define the \textbf{Fourier transform}
    $ \tau : \mathcal{C}(X) \rightarrow \mathcal{C}(X^*) $
    as the semilinear map defined by
    \begin{equation*}
        \begin{split}
            (P_x)^{\tau} &:= \sum_{ \chi\in X^* } \chi(x)P_{\chi} \\
            (\sum_{ x \in X }\alpha_xP_x)^{\tau} &:= \sum_{x \in X}
                \overline{\alpha}_x(P_{x})^{\tau}
        \end{split}
    \end{equation*}
\end{definition}

Note that, as we saw in \ref{lm:bm_subalg}, $ \mathbb{C}(\mathcal{A}) $ is a subset
of $ \mathcal{C}(x)$, so we may apply $ \tau $ to matrices in
$ \mathbb{C}(\mathcal{A}) $.

\begin{proposition}
    For all $ A,B \in \mathcal{C}(X) $ and all $ \alpha \in \mathbb{C} $, it holds
    that
    \begin{enumerate}[label=(\roman*)]
        \item $ (A + B)^{\tau} = A^{\tau} + B^{\tau} $,
            $ (\alpha A)^{\tau} = \overline{\alpha}A^{\tau} $,
        \item $ (AB)^{\tau} = A^{\tau} \circ B^{\tau} $,
            $ (A \circ B)^{\tau} = |X|^{-1} A^{\tau} B^{\tau} $,
        \item $ A^{\tau\tau} = |X|A $,
        \item $ I^{\tau} = J $, $ J^{\tau} = |X|I $.
    \end{enumerate}
    \label{prop:gamma_properties}
\end{proposition}
\begin{proof}
    The first property is obvious from the definition of the Fourier transform.
    We will suppose $ A = \sum_{ x \in X } \alpha_x P_x $ and
    $ B = \sum_{ y \in X } \beta_y P_y $.
    \begin{enumerate}[label=(\roman*)]
        \setcounter{enumi}{1}
        \item We first prove that $ (P_xP_y)^{\tau} = P_x^{\tau} \circ P_y^{\tau} $.
        \[ (P_xP_y)^{\tau} = P_{x+y}^{\tau} = \sum_{ \chi \in X^* }
            \chi(x+y) P_{\chi} = \sum_{ \chi \in X^* } \chi(x)\chi(y) P_{\chi}
            = P_x \circ P_y
        \]
        From this it follows that
        \[
            \begin{split}
            (AB)^{\tau} = \sum_{ x \in X } \sum_{ y \in X } \overline{\alpha_x}
            \overline{\beta_y} (P_xP_y)^{\tau} =
            \sum_{ x \in X } \sum_{ y \in X } \overline{\alpha_x}
            \overline{\beta_y} P_x^{\tau} \circ P_y^{\tau} &=
            \left ( \sum_{ x \in X } \alpha_x P_x \right )^{\tau} \circ
            \left ( \sum_{ y \in X } \beta_y P_y \right )^{\tau}  \\
                        &=A^{\tau} \circ B^{\tau}
            \end{split}
        \]
        We now show that $ (A \circ B)^{\tau} = |X|^{-1}A^{\tau}B^{\tau} $.
        We first prove that $ (P_x \circ P_y)^{\tau} = |X|^{-1}
        P_x^{\tau}P_y^{\tau} $.

        Suppose $ x=y $. Since $ P_x \circ P_x = P_x $, we have that
        \[
            P_x^{\tau} P_x^{\tau} = \sum_{ \chi \in X^* } \sum_{ \eta \in X^* }
            \chi(x) \eta(x) P_{\chi} P_{\eta} =
            \sum_{ \chi \in X^* } \sum_{ \eta \in X^* }
            (\chi\eta)(x) P_{\chi\eta} =
            |X| \sum_{ \chi \in X^* } \chi(x) P_{\chi} = |X| P_x \circ P_x
        \]

        Now suppose $ x \neq y $. Then, since $ P_x \circ P_y = 0 $
        \[
            P_x^{\tau} P_y^{\tau} = \sum_{ \chi \in X^* } \sum_{ \eta \in X^* }
            \chi(x) \eta(y) P_{\chi} P_{\eta} =
            \sum_{ \chi \in X^* } \sum_{ \eta \in X^* }
            (\chi\eta)(x) P_{\chi\eta} =
            |X| \sum_{ \chi \in X^* } \chi(x) P_{\chi} = |X| P_x \circ P_x
        \]
    \end{enumerate}
\end{proof}

We can now prove that the dual scheme is a translation association scheme.

\begin{theorem}[Theorem 2.2.10, \cite{brouwer_distance-regular_1989}]
    For any $ d $-class translation association scheme $ (X,\mathcal{R}) $, with
    eigenmatrices $ P,Q $, the dual scheme $ (X^*, \mathcal{R}) $ is a $ d $-class
    translation associaion scheme with eigenmatrices $ P^* = Q $, $ Q^* = P $.

    Furthermore, it holds that
    \begin{equation}
        Q_{ij} = \sum_{ \chi \in N_j^* } \chi(x) \text{ for any } x \in N_i
        \label{eq:qval_sum_char}
    \end{equation}
    and that
    \begin{equation}
        E_j = |X|^{-1} \sum_{ x \in X } \sum_{ \chi \in N_j^* } \chi(x)P_x
        \label{eq:Ej_perm_mat_sum}
    \end{equation}
    \label{th:tas_dual}
\end{theorem}
\begin{proof}

    We start by showing that $ A_i^* = E_i^{\tau} $ for all
    $ i \in \{ 0,1,\dots,d \} $. Let us fix any $ (\chi,\psi) \in R_k^* $ and
    call $ \nu = \chi^{-1}\psi \in N_k^* $. Then, applying first \ref{lm:Ai_as_sum_of_perm}
    and then \eqref{eq:pval_char_sum}, we get that
    \begin{equation*}
        \begin{split}
            (E_i^{\tau})_{\chi\psi} &=
            |X|^{-1}\sum_{ j=0 }^{ d } \overline{q_i(j)}(A_j^{\tau})_{\chi\psi} =
            |X|^{-1} \sum_{ j=0 }^{ d }\overline{q_i(j)}
                \sum_{ x \in N_j }(P_x^{\tau})_{\chi\psi} = \\
            &= |X|^{-1} \sum_{ j=0 }^{ d }\overline{q_i(j)}
                \sum_{ x \in N_j }(P_x^{\tau})_{\chi\psi}
            = |X|^{-1} \sum_{ j=0 }^{ d }\overline{q_i(j)}\sum_{ x \in N_j }
            \sum_{ \eta \in X^* }\overline{\eta(x)}(P_{\eta})_{\chi\psi} = \\
            &= |X|^{-1} \sum_{ j=0 }^{ d }\overline{q_i(j)}
            \sum_{ l=0 }^{ d } \sum_{ \eta \in X_l^* }(P_{\eta})_{\chi\psi}
                \sum_{ x \in N_j } \overline{\eta(x)}
            = |X|^{-1} \sum_{ j=0 }^{ d }\overline{q_i(j)}
            \sum_{ l=0 }^{ d } \sum_{ \eta \in X_l^* } \overline{p_{j}(l)}
                (P_{\eta})_{\chi\psi}
        \end{split}
    \end{equation*}
    Note that $ (P_{\nu})_{\chi\psi} $ is $ 1 $ since $ \nu = \chi^{-1}\psi =
    \psi\chi^{-1} \iff \psi = \nu\chi $. Applying this and
    \ref{prop:association_schemes_pq_identity}, we get that
    \[
        (E_i^{\tau})_{\chi\psi} = |X|^{-1}\overline{\sum_{ j=0 }^{ d }
        q_i(j)p_{j}(k)} = \delta_{ik}
    \]
    Therefore, $ (E_i^{\tau})_{\chi,\psi} $ is $ 1 $ if
    $ (\chi,\psi) \in R_i^* $ and 0 otherwise, which implies $ E_i^{\tau} = A_i^* $.

    We have that $ (\mathbb{C}(\mathcal{A}))^{\tau} =
    \langle A_0^*, A_1^*, \dots, A_d^* \rangle_{\mathbb{C}} $. We will now show
    that $ X^* $ holds the properties of a translation association scheme:
    \begin{enumerate}[label=(\roman*)]
        \item To show that $ R^* = \{ R_0, R_1, \dots, R_d \} $ partitions
            $ X^* \times X^* $, we note that every $ \nu \in X^* $ is an
            eigenvector of $ A_j $, which implies that it must be in the column
            space of some $ E_i $. It is easy to check that the column spaces of
            idempotent matrices give eigenvectors of eigenvalue $ 1 $.
        \item Since $ E_0 = |X|^{-1}J $, it is clear that $ E_0 v = v $ if and
            only if $ v = 0 $, hence $ R^*_0 = \{ (\chi,\chi) \mid \chi \in X^* \} $.
        \item Note that $ \psi^{-1}\chi = \overline{\chi^{-1}\psi} $. Therefore,
            if $ (\chi,\psi) \in R_i^* $, that is, $ \chi^{-1}\psi $ is in the
            column space of $ E_i $, it is clear that $ \psi^{-1}\chi $ must
            also be in the column space of $ E_i $, and thus
            $ (\psi,\chi) \in R_i^* $.
        \item From \ref{prop:gamma_properties}, we get that
            \[ A_i^* A_j^* = E_i^{\tau}E_j^{\tau} = |X|(E_i \circ E_j)^{\tau} \]
            which is in $ \mathbb{C}(\mathcal{A}^*) = \mathbb{C}(\mathcal{A})^{\tau} $
            since, by \ref{th:idemp_mat_properties}, $ E_j \circ E_i $ is in
            $ \mathbb{C}(\mathcal{A}) $.
        \item Let us fix $ (\chi,\psi) \in R_i^* $. For any $ \eta \in X^* $,
        \[
            E_i((\chi\eta)^{-1}\psi\eta) = E_i(\chi^{-1}\psi) = \chi^{-1}\psi =
            \eta^{-1}\chi^{-1}\psi\eta = (\eta\chi)^{-1}(\psi\eta)
        \]
        and so $ (\chi\eta, \psi\eta) \in R_i^* $.
    \end{enumerate}
    This proves that $ X^* $ is, in fact, a translation association scheme.

    Since we can identify $ X^{**} $ with $ X $, it must now also hold that
    \[ A_i = A_i^{**} = (E_i^*)^{\tau}  \]
    Thus,
    \[
        |X| E_i^* = ((E_i^*)^{\tau})^{\tau} = A_i^{\tau}
        \left ( \sum_{ j=0 }^{ d } p_i(j) E_j \right )^{\tau} =
        \sum_{ j=0 }^{ d } \overline{p_i(j)}E_j^{\tau} =
        \sum_{ j=0 }^{ d } p_i(j) A_i^*
    \]
    and so $ Q^* = P $. Furthermore, $ Q = Q^{**} = P^* $.

    It is now clear that equation \eqref{eq:qval_sum_char} follows from
    \eqref{eq:pval_char_sum}. We may concude that
    \[
        \begin{split}
        E_j = |X|^{-1} \sum_{ i=0 }^{ d } q_j(i)A_i = |X|^{-1}
        |X|^{-1} \sum_{ i=0 }^{ d } Q_{ij}A_i &=
        |X|^{-1} \sum_{ i=0 }^{ d } \left ( \sum_{ \chi \in N_j^*} \chi(x) \right )
            \left ( \sum_{ x \in N_i} P_x \right ) = \\
        &= |X|^{-1} \sum_{ x \in X } P_x \left ( \sum_{ \chi \in N_j^*} \chi(x) \right )
        \end{split}
    \]
\end{proof}

In the context of additive codes in translation association schemes, the concept
of duality takes on a different meaning:
\begin{definition}
    Let $ Y $ be an additive code of the translation association scheme
    $ (X,\mathcal{R}) $ . We define the \textbf{dual code} $ Y' $ of $ Y $ as the
    subgroup
    \[ Y' = \{ \chi \in X^* \mid \chi(y) = 1 \text{ for all } y \in Y \} =
    \{ y' \in X \mid  y \cdot y' = 1 \text{ for all } y \in Y \}\]
\end{definition}

The dual code of $ Y $ is defined in such a way
that their inner distribution is determined by the inner distribution of their
dual, namely:
\begin{theorem}[{\parencite[Theorem 2.10.12]{brouwer_distance-regular_1989}}]
    Let $ Y $ be an additive code $ Y $ of a translation association
    scheme $ (X, \mathcal{R}) $. Then, the inner distributions $ a,a^* $ of
    $ Y $ and its dual code $ Y' $ are related by the formulae
    \begin{equation*}
        a' = \frac{1}{|Y|}aQ, \quad a = \frac{|Y|}{|X|}a'P
    \end{equation*}
    \label{th:add_tas_inner_dist}
\end{theorem}
\begin{proof}
    We start by proving that for translation association schemes the inner
    distribution can actually be expressed as $ a_i = |N_i^* \cap Y| $:
    for every $ y \in N_i^* \cap Y $, every $ z \in Y $ gives a different
    $ (z,y+z) $ in $ (Y \times Y) \cap R_i $.

    Now, to prove the formulae above, we fix a $ k \in \{ 0,1,\dots,d \} $ and
    develop
    \[ |Y|^2a_k = |Y|^2 |Y'\cap N_k^*| \]
    Note that
    from \ref{prop:sum_char_tas} we get that the previous expression above is
    equal to
    \[
        \sum_{ \nu \in N_k^* }\sum_{ y\in Y } |Y|\nu(y) =
        \sum_{ \nu \in N_k^* } \sum_{ y \in Y } \nu(Y) (\sum_{ z \in Y } \nu(-z)) =
        \sum_{ \nu \in N_k^* } \sum_{ y,z \in Y } \nu(Y-z)
    \]
    From \ref{th:tas_dual}, we get that this is equal to
    \[ |X| \boldsymbol{x_Y}^TE_k\boldsymbol{x_Y} \]
    and, from \ref{lm:in_out_dist}, this equals
    \[ |Y|(aQ)_k \]

\end{proof}
This last proposition, when applied to the translation association scheme obtained
over matrices with the rank distance, gives an analog of the
\textit{MacWilliams identities} established for classical codes.

\newpage
\printbibliography

\end{document}
